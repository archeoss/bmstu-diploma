\specsection{ЗАКЛЮЧЕНИЕ}

% \addcontentsline{toc}{specsection}{ЗАКЛЮЧЕНИЕ}
Сравнение рассмотренных методов будет проводиться по следующим критериям:
\begin{itemize}
    \item Пакетная обработка --- характеристика способа обработки данных в виде пакетов (при присутствии оной);
    \item Потоковая обработка --- характеристика способа обработки данных в виде потоков (при присутствии оной);
    \item Механизм коммуникации --- характеристика способа передачи данных между узлами в системе;
    \item Модель вычисления --- оценка программной модели, отвечающей за обработку данных;
    \item Планирование задач и балансировка нагрузки --- характеристика алгоритмов планирования и балансировки нагрузки;
    \item Масштабируемость --- качественная оценка возможностей системы масштабировать вычисления на весь кластер;
    \item Отказоустойчивость --- качественная оценка возможностей системы восстановливаться после сбоев или неисправностей;
    % \item Сценарий использования --- оценка вариантов использования, для которых лучше всего подходит данная система;
\end{itemize}

На таблицах \ref{tab:class_models} -- \ref{tab:class_models4} приведена классификация моделей по обозначенным выше критериям.

\begin{table}[H]\centering
	\captionsetup{justification=raggedright,singlelinecheck=off}
	\captionof{table}{Классификация програмных моделей по способам пакетной и потоковой обработке}\label{tab:class_models}
	\begin{tabular}{|c|c|c|}
		\hline
		\bfseries Модель & \bfseries Пакетная обработка
		                 & \bfseries Потоковая обработка \\
		\hline
		MapReduce       & \makecell{Пакеты \\ фиксированного размера.} 
		                & \makecell{Серия малых пакетов.} \\
		\hline
		Apache Spark    & \makecell{Обработка \\ посредством \texttt{RDD}}
		                & \makecell{Серия малых пакетов.} \\
		\hline
		Pregel          & \makecell{Итеративный алгоритм \\ для обработки графов.}
		                & \makecell{Отсутствует.} \\
		\hline
		DryadLinq       & \makecell{Формирует пакеты\\ исходя из\\ \texttt{LINQ} выражений.}
		                & \makecell{Потоковая обработка\\ ранее сформированных\\ пакетов.} \\
		\hline
		Google Dataflow & \makecell{Оконная обработка пакетов.} 
		                & \makecell{Обрабатка\\ в реальном времени.} \\
		\hline
		Apache Flink    & \makecell{Оконная обработка пакетов.}
                        & \makecell{Обрабатка\\в реальном времени.} \\
		\hline
	\end{tabular}
\end{table}

\begin{table}[H]\centering
	\captionsetup{justification=raggedright,singlelinecheck=off}
	\captionof{table}{Классификация программных моделей по механизмам коммуникации и планированию задач}\label{tab:class_models2}
	\begin{tabular}{|c|c|c|}
		\hline
		\bfseries Модель & \bfseries Механизм коммуникации
		                 & \makecell{\bfseries Планирование задач \\ \bfseries и\\ \bfseries балансировка нагрузки} \\
		\hline
		MapReduce       & \makecell{Мастер-рабочий\\ архитектура} 
		                & \makecell{Мастер отвечает\\ за балансировку} \\
		\hline
		Apache Spark    & \makecell{Мастер-рабочий\\ архитектура}  
		                & \makecell{Направленный ациклический \\ граф (DAG) \\ из задач} \\
		\hline
		Pregel          & \makecell{Передеча сообщений \\ между вершинами графа} 
		                & \makecell{Разбиение графа \\ мастер узлом} \\
		\hline
		DryadLinq       & \makecell{Мастер-рабочий\\ архитектура}
		                & \makecell{Направленный ациклический \\ граф (DAG) \\ из задач} \\
		\hline
		Google Dataflow & \makecell{Коммуникация через \\ конвейеры}
		                & \makecell{Оптимизацияя плана \\ выполнения на основе \\ графа Dataflow} \\
		\hline
		Apache Flink    & \makecell{Поточная передачи данных} 
		                & \makecell{Распределение буфера \\ между потребителями} \\
		\hline
	\end{tabular}
\end{table}

\begin{table}[H]\centering
	\captionsetup{justification=raggedright,singlelinecheck=off}
	\captionof{table}{Классификация программных моделей по модели вычисления}\label{tab:class_models3}
	\begin{tabular}{|c|c|}
		\hline
		\bfseries Модель    & \bfseries Модель вычисления \\
		\hline
		MapReduce           & \makecell{Двухэтапная вычислительная модель} \\
		\hline
		Apache Spark        & \makecell{Манипуляция\\ устойчивыми распределенными\\ наборами данных} \\
		\hline
		Pregel              & \makecell{Вершинно-центричная модель} \\
		\hline
		DryadLinq           & \makecell{Декларативная модель} \\
		\hline
		Google Dataflow     & \makecell{Унифицированная модель\\ для пакетной и\\ потоковой обработки}. \\
		\hline
		Apache Flink        & \makecell{Обработка потоков} \\
		\hline
	\end{tabular}
\end{table}

\begin{table}[H]\centering
	\captionsetup{justification=raggedright,singlelinecheck=off}
	\captionof{table}{Классификация программных моделей по масштабируемости и устойчивости к сбоям}\label{tab:class_models4}
	\begin{tabular}{|c|c|c|}
		\hline
		\bfseries Модель            & \bfseries Масштабируемость 
		                            & \bfseries Устойчивость к сбоям \\
		\hline
		MapReduce                   & \makecell{Горизонтально масштабируема} 
		                            & \makecell{Повторное выполнение \\ проваленных задач}. \\
		\hline
		\makecell{Apache\\ Spark}   & \makecell{Горизонтально масштабируема\\ с использованием\\ оперативной памяти}. 
		                            & \makecell{Репликация данных\\ в рамках RDD} \\
		\hline
		Pregel                      & \makecell{Масштабируется в рамках \\ обработки графов} 
		                            & \makecell{Точки контроля \\ и восстановление \\ состояния} \\
		\hline
		DryadLinq                   & \makecell{Горизонтально масштабируема} 
		                            & \makecell{Точки контроля \\ и восстановление \\ состояния} \\
		\hline
		\makecell{Google \\ Dataflow}   & \makecell{Динамическая масштабируемость \\ под размер данных} 
		                                & \makecell{Точки контроля \\ и повторное \\ выполнение} \\
		\hline
		Apache Flink                & \makecell{Масштабируется \\ путем добавления \\ менеджеров задач}
		                            & \makecell{Точки контроля \\ и распределенние\\ снимов состояния} \\
		\hline
	\end{tabular}
\end{table}


Результаты, полученные в результате анализа упомянутых систем, приводят к следующим ключевым выводам:
% \begin{itemize}
%     \item Масштабируемость:
%
%         Все системы уделяют особое внимание горизонтальной масштабируемости, что позволяет им эффективно справляться с растущими рабочими нагрузками за счет добавления большего количества узлов в кластер.
%         Горизонтальная масштабируемость является ключевым аспектом для обработки больших наборов данных и сложных вычислений, и каждая система предоставляет механизмы для достижения этой цели.
%
%     \item Отказоустойчивость:
%     
%     \item Модели вычисления:
%         Модели вычисления этих систем различаются, учитывая различные варианты использования и предпочтения разработчиков.
%         MapReduce следует двухэтапной модели с функциями Map и Reduce, в то время как Apache Spark предоставляет высокоуровневую, выразительную модель программирования с поддержкой различных API.
%         Pregel фокусируется на вершинно-ориентированной модели программирования, адаптированной для обработки графов, в то время как DryadLINQ использует декларативную модель программирования с использованием синтаксиса LINQ.
%         Google Dataflow и Apache Flink предлагают унифицированные модели программирования как для пакетной, так и для потоковой обработки, обеспечивая гибкость и согласованность при разработке приложений.
%
%     \item Механизмы коммуникации:
%         Механизмы взаимодействия, используемые этими системами, различаются в зависимости от целей их проектирования и предполагаемых вариантов использования.
%         Такие системы, как MapReduce и Apache Flink, используют модель взаимодействия между мастером и работником, в то время как Pregel использует связь, ориентированную на вершины, для обработки графов.
%         Apache Spark и Google Dataflow используют модели потоков данных для обмена данными, а DryadLINQ использует каналы передачи данных для соединения вершин в своих графах потоков данных.
%
%     \item Варианты использования и гибкость:
%         Каждая система предназначена для конкретных случаев использования, и их гибкость в обработке различных типов рабочих нагрузок способствует их широкому внедрению.
%         Поддержка Apache Spark как пакетной, так и потоковой обработки делает ее универсальной, в то время как акцент Pregel на графовых алгоритмах отвечает специализированным требованиям.
%         Унифицированная модель Google Dataflow поддерживает как ограниченные, так и неограниченные наборы данных, демонстрируя свою адаптивность. Apache Flink отличается низкой задержкой обработки потоков, что подчеркивает его пригодность для аналитики в реальном времени.
%
% \end{itemize}
\begin{itemize}
    \item Хотя основополагающая концепция горизонтальной масштабируемости остается неизменной во всех изученных моделях, подходы к ее достижению демонстрируют адаптацию к современным вызовам.

    \item Отказоустойчивость становится краеугольным камнем при проектировании этих систем, отражая присущей распределенным средам непредсказуемости.
        Помимо традиционных механизмов восстановления после сбоев, интеграция информации о происхождении, автоматическое определение контрольных точек и репликация данных демонстрируют приверженность обеспечению целостности данных и последовательной обработки.

    \item Разнообразие моделей вычисления отражает понимание разнообразных требований, предъявляемых различными сценариями обработки данных.
        Apache Flink и Google Dataflow выделяются тем, что предлагают унифицированные модели программирования как для пакетной, так и для потоковой обработки, облегчая когнитивную нагрузку на разработчиков, которым поручено создавать универсальные приложения для обработки данных.

    \item Коммуникационные модели раскрывают архитектурные тонкости, предназначенные для оптимизации потока данных в этих системах.
            От модели <<мастер-рабочий>> в MapReduce до подхода, основанного на потоке данных в Google Dataflow, коммуникационная парадигма каждой системы соответствует предполагаемым вариантам использования, демонстрируя нюансы решений, принимаемых для повышения эффективности и быстродействия.

    \item Возможности систем для решения специализированных задач подчеркивают их преимущества в своей нише. Сосредоточенность Pregel на обработке графов и специализация Apache Flink в обработке потоков с низкой задержкой подчеркивают индивидуальные решения, предлагаемые этими системами.
        Адаптивность Google Dataflow для обработки как ограниченных, так и неограниченных наборов данных подчеркивает его универсальность, способствуя его актуальности в широком спектре сценариев обработки данных.
\end{itemize}
В заключение следует отметить, что эти распределенные системы обработки данных служат не только инструментами для управления крупномасштабными данными, но и артефактами, отражающими нюансы решений, принятых их разработчиками. Они демонстрируют динамичный отклик на меняющийся ландшафт распределенных вычислений, где адаптивность, отказоустойчивость и продуманные парадигмы программирования объединяются, чтобы дать разработчикам возможность решать разнообразные задачи обработки данных.

\clearpage
