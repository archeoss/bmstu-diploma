\specsection{ВВЕДЕНИЕ}
% \addcontentsline{toc}{specsection}{Введение}

Компьютерная индустрия изменила курс в 2005 году, когда \texttt{Intel}, последовав примеру \texttt{IBM Power} 4 и процессору \texttt{Niagara} от \texttt{Sun Microsystems}, объявили, что их высокопроизводительные микропроцессоры отныне будут опираться на несколько процессоров или ядер.
Новое в отрасли слово <<многоядерный>> отражает план удвоения количества стандартных ядер на матрицу с каждым поколением полупроводниковых процессов.
Многоядерный процессор, очевидно, поможет многопрограммным рабочим нагрузкам, которые содержат набор независимых последовательных задач, но как отдельные задачи станут быстрее?
Переход от последовательных вычислений к умеренно параллельным значительно усложняет программирование, не вознаграждая эти большие усилия значительно лучшим соотношением производительности к энергопотреблению.
Следовательно, многоядерные процессоры едва ли являются идеальным решением.
Подкрасться к проблеме параллелизма с помощью многоядерных решений, скорее всего, не удастся, и нам отчаянно нужно новое решение для параллельного аппаратного и программного обеспечения.
Гипотеза заключается не в том, что традиционные научные вычисления - это будущее
параллельных вычислений; она заключается в том, что совокупность знаний, полученных при создании программ, которые
хорошо работают на массово параллельных компьютерах, может оказаться полезной при распараллеливании будущих
приложений.

