% Контекст: это Научно Исследовательская Работа. Цель работы - анализ методов распределенных вычислений в распределенных системах хранения информации.

\section{Аналитический раздел}

В данном разделе будут рассмотрены основные понятия и термины, которые используются в данной работе, а также будут рассмотрены основные алгоритмы и существующие решения. 

\subsection{Проблематика}

Различные сетевые приложения, такие как веб-серверы, поисковые системы, системы управления базами данных и т.д. стали неотъемлемой частью современного общества. 
Об этом свидетельствует рост количества пользователей и объемов данных, которые обрабатываются этими приложениями, а также общей долей рынка связанной с этими приложениями~\cite{iot-market}.
Масштаб современных сетевых приложений предоставляет конечному пользователю дополнительные вычислительные мощности для решения прикладных задач и распределяет общую нагрузку на сеть посредством распределения вычислений между узлами сети.
Когда компьютеры работают вместе в рамках единой сети, мощность всех подключенных к сети компьютеров может использоваться для выполнения сложных задач.
Подобного рода вычисления могут быть распределены на централизованные и распределенные.

Централизованное решение основано на том, что один узел назначается ответственным за все вычисления проводимые приложением, обрабатывает их локально, и данный узел является общим для всех пользователей системы.
Следовательно, существует единая точка контроля и единая точка отказа. 
Возможно такого рода решение и является оптимальным для небольших задач, но при увеличении нагрузки на систему, возникает, например, необходимость в более отказоустойчивой системе, которая будет обеспечивать доступность данных на протяжении всего времени работы системы.

Мотивацией роста децентрализованных вычислений является доступность недорогих, высокопроизводительных компьютеров и сетевых инструментов. 
Такая система может обладать более высокой производительностью, чем один конкретный суперкомпьютер. 
Целью таких систем является минимизация затрат на связь и вычисления. 
В распределенных системах этапы обработки приложения распределены между участвующими в ней узлами. 
Основным шагом во всех архитектурах распределенных вычислений является понятие связи между узлами системы.

Приложение, удовлетворяющему этим требованиям, обычно строится из стандартных <<блоков>>, которые предоставлют необходимую функциональность.
Так например, приложение может требовать реализации следующих функций:
\begin{itemize}
  \item[$-$] Хранение данных для дальнейшего к ним доступа (базы данных);
  \item[$-$] Хранение результата дорогостоящей операции для ускорения чтения (кэширование);
  \item[$-$] Возможность для пользователей искать данные по ключевому слову или фильтровать их различными способами (поисковые индексы);
  \item[$-$] Отправление сообщений другому процессу, которые будут обрабатываться асинхронно (потоковая обработка);
  \item[$-$] Периодическая обработка большого объема накопленных данных (пакетная обработка).
\end{itemize}

Распределенная система --- это приложение, которое выполняет набор протоколов для координации действий нескольких процессов в сети таким образом, что все компоненты взаимодействуют друг с другом для выполнения одной задачи или небольшого набора взаимосвязанных задач~\cite{thampi2009introduction}.
Сотрудничающие компьютеры могут получать доступ как к удаленным, так и к локальным ресурсам данной распределенной системы.
Существование нескольких автономных компьютеров в распределенной системе неочевидно для пользователя, т.е. пользователь не знает, что задания выполняются несколькими компьютерами в рамках единой системы. 

Как и к любому другому ПО, к распределенным системам применимы часто рассматриваемые инженерные критерии~\cite{kleppmann2017designing}, а именно:
\begin{enumerate}
  \item \texttt{Reliability} --- Надежность --- Система должна продолжать корректно работать (выдавать корректный результат на
желаемом уровне производительности) даже перед лицом неблагоприятных факторов (аппаратных или
программных сбоев и возможной человеческой ошибки); 
  \item \texttt{Scalability} --- Масштабируемость --- По мере роста системы (при увеличении объема данных, трафика или сложности) должны существовать разумные способы и инструменты управления этим ростом;
  \item \texttt{Maintainability} --- Сопровождаемость --- Со временем, набор разработчиков, работающих над системой, может сильно изменяться, и все они должны быть в состоянии продуктивно работать над продуктом (проектирование и эксплуатация, как поддержание текущего поведения, так и адаптация системы к новым требованиям). 
\end{enumerate}
В дальнейшем описываемые алгоритмы будут расматриваться через призму этих критериев: как результирующие системы удовлетворяют этим критериям и какие компромиссы приходится делать при их реализации.

Также по мере разработки распределенных систем, существует набор различных вопросов, на которые необходимо дать ответ для корректной и эффективной работы системы как таковой.
Например, как распределить задачи между узлами системы и как обеспечить надежность и целостность данных, 
как гарантировать доступность данных на всем протяжении работы системы и т.д.
Одним из формальных ответов на эти вопросы служит формулировка \texttt{CAP}-теоремы --- или теорема Брюэра --- которая звучит следующим образом: 

В любой распределенной системе можно обеспечить не более двух из трех свойств:
\begin{itemize}
  \item[$-$] \textbf{C}onsistency --- Согласованность --- каждый сервер возвращает верный ответ на каждый запрос, т.е. ответ, который является правильным в соответствии с требуемой спецификацией приложения;
  \item[$-$] \textbf{A}vailability --- Доступность --- гарантируется, что на каждый запрос в конечном итоге будет получен некий ответ;
  \item[$-$] \textbf{P}artition tolerance --- Устойчивость к разделению --- в произвольный момент времени сервера могут быть разделены на несколько групп, которые не будут взаимодействовать друг с другом. Иными словами, сообщения в такой системе могут задерживаться, а иногда и теряться.
\end{itemize}

К сожалению, такая формулировка вводит в заблуждение~\cite{cap-changed}, поскольку в случае если система существует в сети, которая может терять произвольное количество сообщений, что является реальностью современных телекоммуникаций, то такая система не может быть одновременно доступной и согласованной, выбрать следует что-то одно~\cite{cap-confusion}.
Поэтому в дальнейшем, при рассмотрении распределенных систем, будут браться во внимание именно эти два критерия: доступность и согласованность, принимая устойчивость к разделению как данность.
Это необходимое допущение позволяет лучше понять проблематику распределенных вычислений и рассматривать их в контексте реальных систем.

\subsection{Основные алгоритмы}
todo
\subsubsection{MapReduce}
todo
\subsubsection{Graph traversals}
todo
\subsubsection{Finite State Machine}
todo
\subsubsection{Dynamic Programming}
todo
\subsubsection{Lambda architecture}
todo

\subsection{Существующие решения}
todo
\subsubsection{Hadoop}
todo
\subsubsection{Spark}
todo
\subsubsection{Flink}
todo
\subsubsection{Hive}
todo
\subsubsection{Databricks}
todo
\subsubsection{GraphX}
todo

